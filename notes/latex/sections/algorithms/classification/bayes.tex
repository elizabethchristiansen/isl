\subsection{The Bayes Classifier}

The Bayes classifier is an ideal classifier that always places an unknown observation in the class for which is has the highest probability of membership. It assigns to a class $j$ for which

$$ Pr(Y=j|X=x_{0}) $$

is maximized. Where $Y$ is the observation's unknown class and $x_{0}$ is the observations predictor values. Simple enough, but actually calculating this probability is usually not directly possible and we therefore either need to approximate it or use another method entirely.

We can use this to define a Bayes \textbf{decision boundary} where we can decide an observation's class by which side of the boundary it falls on.

The Bayes classifier also gives rise to an ideal, irreducible error rate which is the theoretical best performance any classification algorithm could obtain given the data available:

$$ 1 - E[max_{j} Pr(Y=j|X)] $$

Where $max_{j}$ is the value of $j$ which maximizes $Pr(...)$. The expectation is the average over all possible values of $X$. This error measure can be read as the error when all observations (all values of $X$) are assigned to their highest probability class which, if we knew the conditional distribution $Pr(Y|X)$, would be exactly the error we would obtain when using the Bayes classifier and since the Bayes classifier is ideal so to is the error (it is irreducible).
