Clustering refers to a set of methods aimed at detecting subgroups within data - points that are collectively 'closer' (with whatever measure you use to determine closeness) to each other than to other points/subgroups. In general, clustering determines \textbf{distinct} subgroups (no overlaps) and thus each observation belongs to exactly one subgroup. There are advanced methods which potentially allow for joint association or no association. Clustering can be performed either by clustering the features based on the observations (what features do similar observations posses) or by clustering the observations based on the features (what observations share similar features). For a dataset consisting of a set of users who buy certain items from a website, clustering on the features could result in subgroups of users (observations) who buy similar items e.g. a subgroup of users who all buy technology goods. Clustering on the observations could result in subgroups of items (features) bought by similar users e.g. a subgroup of expensive items bought by users with a lot of disposable income.
